\documentclass[20pt,landscape]{foils}
\usepackage[english,german]{babel}  % language support for german/english
\usepackage[latin1]{inputenc}       % allow Latin1 characters
\usepackage{supertabular}
\usepackage{ifvtex}
\usepackage{ifpdf}
% Running vtex we most probably also create pdf...
\ifvtexpdf\pdftrue\fi
\ifpdf
\usepackage{pause}               % loads also color.sty
\usepackage{background}
\usepackage{graphicx}            % for including graphics
\usepackage{geometry}
\usepackage{hyperref}
\else
\usepackage[dvipdfm]{pause}      % loads also color.sty
\usepackage[dvipdfm]{background}
\usepackage[dvips]{graphicx}
\usepackage[dvips]{geometry}
\usepackage[dvips]{epsfig}
\usepackage[dvipdfm]{hyperref}

% Math packages
\usepackage[fleqn]{amsmath}
\usepackage{amssymb}
\usepackage{euler}
\usepackage{threeparttable}
\usepackage{url}
\usepackage[all]{xy}        
\usepackage{dcolumn}
\newcolumntype{.}{D{.}{.}{-1}}
\newcolumntype{d}[1]{D{.}{.}{#1}}
\rightfooter{}


%% The following definition is only necessary, because we use variants
%% of the same graphics file in .mps and .eps format each. The variant
%% .mps is suitable for pdflatex and dvipdfm, the variant .eps is good
%% for vlatex and pdflatex. Therefore we have to setup a preference of
%% .mps for dvipdfm.
%% matrixb?e.eps is created from matrixb?e.fig using the option -p2 of
%% fig2dev. matrixb?e.mps is created running fig2dev without such an
%% option (see the manual for PPower4).
\DeclareGraphicsExtensions{.jpg,.jpeg,.pdf,.png,.mps,.eps,.ps}
\fi
\usepackage{pp4slide}
\geometry{headsep=3ex,hscale=0.9}
\hypersetup{pdftitle={Predictive algortihms are stable},
  pdfsubject={MTNS},
  pdfauthor={Sayan Mukherjee, Center for Biological and Computational
    Learning, MIT, Center for Genome Research, 
    Whitehead Institute,  sayan@mit.edu},
  pdfpagemode={FullScreen},
  linkcolor=cyan,               
  citecolor=cyan,               
  pagecolor=cyan,
  urlcolor=cyan
    }

%%%%%%%%%%%%%%%%%%%%%%%%%%%%% colors
\usepackage{wcolor}                      % Definitions
\renewcommand\Black{\color{dblue}}       % color of page number (dblue is off)
\renewcommand\Black{\color{dblue}}       % color of page number
\renewcommand\normalcolor{\color{dblue}} % Color of the text of \foilhead{}
%\vpagecolor[dblue]{RGBblack}  %background: blue to black vertical gradient
%\hpagecolor[blue]{RGBblack} %background: blue to black horizontal gradient
\pagecolor{white}            %background: monocrome


%\pausecolors{textcolor}{normalcolor}{HighlightColor} % Out/In focus
\pausecolors{yellow}{darkyellow}{yellow}      % \Ia yellow  
\pausecolors{magenta}{lightmagenta}{magenta}  % \Ib magenta 
\pausecolors{green}{lightgreen}{green}        % \Ic green   
\pausecolors{lightcyan}{cyan}{lightcyan}      % \Id blue    
\pausecolors{red}{lightred}{red}              % \Ie red     
\pausecolors{black}{darkwhite}{white}         % \Iw white   



\newcommand\LOGOOFF{
   \LogoOff
   \rightfooter{\pauselevel{=1 +1} 
                \rlap{\quad\textsf{\tiny\thepage}}}
}
\newcommand{\sP}{{\rm I\!P}}
\renewcommand{\P}{\text{P}}
\newcommand{\R}{{\rm I\!R}}
\newcommand{\pr}{{\rm I\!P}}
\newcommand{\E}{{\rm I\!E}}
\newtheorem{conj}{Conjecture}
\newtheorem{countex}{Counterexample}
\newtheorem{prop}{Proposition}
\newtheorem{thm}{Theorem}
\newtheorem{cor}{Corollary}
\newtheorem{defn}{Definition}
\newtheorem{lem}{Lemma}
\renewcommand{\ell}{l}
\newcommand{\SSS}{\scriptscriptstyle}
\newcommand{\MC}{\multicolumn}
% \Bpara{x coord}{y coord}{rotation angle in degrees}{height in pt}
 \newcommand{\Bpara}[4]{\begin{picture}(0,0)%
     \setlength{\unitlength}{1pt}%
     \put(#1,#2){\rotatebox{#3}{\raisebox{0mm}[0mm][0mm]{%
           \makebox[0mm]{$\left.\rule{0mm}{#4pt}\right\}$}}}}%
     \end{picture}}

% use instead of \foilhead[]{}
%\foilh[opt'l additional space after header]{title of slide and bookmark}
%\newcounter{bookc}
%\newcommand{\foilh}[2][0in]{%
%  \addtocounter{bookc}{1}
%  \hypertarget{\thebookc}{\foilhead{#2}}
%  \pdfbookmark{#2}{\thebookc}} 
%\newcommand{\vs}{\vspace{-\baselineskip}}

\title{\If{Stable Algorithms are Predictive}}

\author{\large Sayan Mukherjee}

\date{
\Ib{Center for Biological and Computational Learning}\\
\Ib{Cancer Genomics Group}\\
\Ib{MIT}\\
\mbox{}\\
\Ic{Departments of Computational Biology and Statistics}\\
\Ic{Duke University}\\ 
sayan@mit.edu \\
\mbox{}\\
{\small joint work with Tomaso Poggio, Partha Niyogi, and Ryan Rifkin.}}


%%% mathdefs.tex
%


% Taken from Adam Lupu-Sax ....................................................
%%% derivatives
\newcommand{\deriv}[2]{\frac{d#1}{d#2}}
\newcommand{\derivc}[3]{\left. \frac{d#1}{d#2}\right|_{#3}}
\newcommand{\pd}[2]{\frac{\partial #1}{\partial #2}}
\newcommand{\pdc}[3]{\left. \frac{\partial #1}{\partial #2}\right|_{#3}}

%%% Dirac notation
\newcommand{\bra}[1]{\left\langle #1\right|}
\newcommand{\ket}[1]{\left|#1\right\rangle}
\newcommand{\braket}[2]{\left\langle #1 \left|#2\right.\right\rangle}
\newcommand{\braOket}[3]{\left\langle #1\left|#2\right|#3\right\rangle}

% calligraphic letters in math.
\def\cal#1{\mathcal{#1}}

% fonts for quantities in principles known (not known but not
% estimated)
 
\def\fp#1{\mathfrak{#1}}

% Taken from M Haggerty ......................................................
\def\avg#1{\left< #1 \right>}
\def\abs#1{\left| #1 \right|}
\def\recip#1{\frac{1}{#1}}
\def\vhat#1{\hat{{\bf #1}}}
\def\smallfrac#1#2{{\textstyle\frac{#1}{#2}}}
\def\smallrecip#1{\smallfrac{1}{#1}}

% SPSmith's definitions ......................................................
\def\spshalf{{1\over{2}}}
\def\Orabi{\Omega_{\rm rabi}}
\def\btt#1{{\tt$\backslash$#1}}




% My own .....................................................................

\newcommand{\bbeta}{{\boldsymbol\beta}}
\newcommand{\btheta}{{\boldsymbol\theta}}
\newcommand{\bPhi}{{\boldsymbol\Phi}}
\newcommand{\bmu}{\boldsymbol{\mu}}
\newcommand{\bSigma}{{\boldsymbol\Sigma}}
\newcommand{\cN}{{\cal N}}
\newcommand{\cP}{{\cal P}}
\newcommand{\bW}{\mathbf{W}}
\newcommand{\bQ}{\mathbf{Q}}
\newcommand{\bZ}{\mathbf{Z}}
\newcommand{\bS}{\mathbf{S}}
\newcommand{\bT}{\mathbf{T}}
\newcommand{\bz}{{\mathbf z}}
\newcommand{\bx}{{\mathbf x}}
\newcommand{\bX}{{\mathbf X}}
\newcommand{\half}{\textstyle{\frac{1}{2}}}
\newcommand{\real}{\mathbb{R}}
\newcommand{\bepsilon}{\boldsymbol{\epsilon}}
\newcommand{\mub}{{\bar \mu}}
\newcommand{\mup}{\overline{\mu^\prime}}
\newcommand{\cK}{{\cal K}}
\newcommand{\cU}{{\cal U}}
\newcommand{\tD}{\text{D}}
\newcommand{\tE}{\text{E}}
\newcommand{\tV}{\text{V}}
\newcommand{\ns}{\mathfrak{N}}


\parindent=0pt
\begin{document}
\color{RGBblack}
%\LOGOOFF
\hypersetup{pdfpagetransition=Replace}
\maketitle

%%%%%%%%%%%%%%%%%%%%%%%%%%%%%%%%%%%%%%%%%%%%%%%%%%%%%%%%%%%%%%%%%%%%%%


\foilhead{Overview}



\begin{enumerate}

\item Background and definitions \pause

\item Empirical risk minimization: Stability $\Leftrightarrow$ generalization $\Leftrightarrow$ consistency \pause

\item Symmetric algorithms: Stability $\Rightarrow$ generalization \pause

\item Discussion 


\end{enumerate}


%%%%%%%%%%%%%%%%%%%%%%%%%%%%%%%%%%%%%%%%%%%%%%%%%%%%%%%%%%%%%%%%%%%%%%


\foilhead{References}

{\small


\begin{enumerate}

\item L.~Devroye and T.~Wagner.
Distribution-free performance bounds for potential function rules. 
{\it IEEE Trans. Inform. Theory}, 25(5):601--604, 1979. \pause

\item M.~Kearns and D.~Ron.
Algorithmic stability and sanity-check bounds for leave-one-out 
cross-validation.{\it Neural Computation}, 11:1427--1453,1999. \pause

\item O.~Bousquet and A.~Elisseeff.
Stability and Generalization. {\it Journal of Machine Learning
  Research}, 2:488--526, 2002. \pause

\item S.~Kutin and P.~Niyogi.
Almost-everywhere stability and generalization error. Technical
report TR-2002-03, University of Chicago, 2002. \pause

\Ib{\item S.~Mukherjee, P.~Niyogi, T.~Poggio, and R.~Rifkin.
Statistical Learning: stability is sufficient for generalization and necessary and sufficient for consistency of Empirical Risk Minimiation. Technical
report TR-2002-024, MIT, 2002.  \pause}


\Ib{\item T.~Poggio, R.~Rifkin, S.~Mukherjee, and P.~Niyogi. General conditions for predictivity in learning theory. {\it Nature}, 428(25):419:422, 2004. }


\end{enumerate}}


%%%%%%%%%%%%%%%%%%%%%%%%%%%%%%%%%%%%%%%%%%%%%%%%%%%%%%%%%%%%%%%%%%%%%%


\foilhead{The learning problem}



Given dataset $S = \{z_1=(x_1,y_1),...,z_n=(x_n,y_n)\}$ drawn i.i.d.
from a distribution $\mu(x,y)$.\\ \\ \pause

An algorithm is the map ${\cal A}: S \rightarrow f_S$ in general
$f_S \in {\cal H}$. \\ \\ \pause

Predictivity: $f_S(x_{\mbox{new}}) \approx y_{\mbox{new}}$. \\ \\ \pause

Loss function: $V(f_S,z) \equiv V(f_S(x),y)$\\ \pause
\centerline{Examples: Square loss $(f_S(x)-y)^2$, \, \, Classification loss 
$I_{(- y f_S(x) > 0)}$.}



%%%%%%%%%%%%%%%%%%%%%%%%%%%%%%%%%%%%%%%%%%%%%%%%%%%%%%%%%%%%%%%%%%%%%%


\foilhead{Two ``independent'' approaches to learning}



\begin{enumerate}

\item Learning algorithms should generalize (motivation for theory). \\ \\ \pause 


\item Learning algorithms should be stable, well-posed (motivation for algorithms). \\ \\ \pause 

\item We will show that $(2) \Rightarrow  (1)$ and for certain algorithms
$(2) \Leftrightarrow  (1)$. 


\end{enumerate}







%%%%%%%%%%%%%%%%%%%%%%%%%%%%%%%%%%%%%%%%%%%%%%%%%%%%%%%%%%%%%%%%%%%%%%

\foilhead{Definitions}


\begin{itemize}

\item Empirical error: $R_{S}[f] = n^{-1}\sum_{i=1}^n V(f(x_i),y_i)$ \\ \pause

\item Expected error: $R[f] = \E_z V(f,z) \equiv \int_{X \times Y} V(f(x),y) d\mu(x,y)$ \\ \pause

\item Empirical risk minimization: ${\cal A}: \, \, \, \, f_{S} \in \arg \min_{f \in {\cal H}} R_{S}[f]$ \\ \pause

\item Best function in the class: $f_{\cal H} \in \arg \min_{f \in {\cal H}} R[f]$ \\ \pause

\item Leave-one-out error: $R_{loo}[S] = n^{-1}\sum_{i=1}^n V(f_{S^i}(x_i),y_i)$.





\end{itemize}



%%%%%%%%%%%%%%%%%%%%%%%%%%%%%%%%%%%%%%%%%%%%%%%%%%%%%%%%%%%%%%%%%%%%%%

\foilhead{Learning algorithms}

{\small

\begin{defn}
Symmetric algorithms: $\cal A$ is symmetric if it is on average 
invariant to the order of samples:  if over training sets $S$
$$\E_S V(f_S,z) = \E_{S,\pi} V(f_{S(\pi)},z),$$ 
where $S(\pi) = \{z_{\pi(1)},...,z_{\pi(n)} \}$. \\ \pause
\end{defn}


\begin{defn}
Empirical risk minimization (ERM): Given $S$ and a function
space $\cal H$ select $f_S$ such that
$$R_S[f_S] = \min_{f \in {\cal H}} R_S[f] .$$
\end{defn} 
}


%%%%%%%%%%%%%%%%%%%%%%%%%%%%%%%%%%%%%%%%%%%%%%%%%%%%%%%%%%%%%%%%%%%%%%

\foilhead{Generalization and consistency}

{\small

\begin{defn} Generalization: An algorithm ${\cal A}$ generalizes
if 
$$\forall \varepsilon > 0 \, \, \lim_{n \rightarrow \infty} \sup_{\mu} \pr\{|R_S[f_S] - R[f_S]| > \varepsilon \} = 0.$$
\end{defn} \pause 


\begin{defn}
Consistency: ERM is consistent if 
$$\forall \varepsilon > 0 \, \, \lim_{n \rightarrow \infty} \sup_{\mu} \pr\left\{R[f_S]> \inf_{f \in {\cal H}} R[f] + \varepsilon \right\} = 0.$$
\end{defn} 


%%%%%%%%%%%%%%%%%%%%%%%%%%%%%%%%%%%%%%%%%%%%%%%%%%%%%%%%%%%%%%%%%%%%%%

\foilhead{Uniform convergence}

{\small

\begin{defn} Uniform Glivenko Cantelli (uGC) classes: A class of
functions $\cal H$ is a uniform Glivenko Cantelli class if 
$$\forall \varepsilon > 0 \, \, \lim_{n \rightarrow \infty} \sup_{\mu} \pr\left\{\sup_{f \in {\cal H}} \left|n^{-1}\sum_{i=1}^n f(x_i) - \E_x f(x)\right| > \varepsilon \right\} = 0.$$
\end{defn}





}

%%%%%%%%%%%%%%%%%%%%%%%%%%%%%%%%%%%%%%%%%%%%%%%%%%%%%%%%%%%%%%%%%%%%%%



\foilhead{Generalization of ERM}

{\small


${\cal G}$ is the composition of the function class 
$\cal H$ and the loss function $V$
$${\cal G} = \{g(z) = V(f,z) \mbox{ for all } f \in {\cal H}, z \in Z\}.$$ \pause \\ \\



\begin{thm} If the loss function is bounded $0 \leq V(f,z) \leq M$ then 
the following are equivalent \\
a) ERM generalizes \\
b) ERM is consistent \\
c) $\cal G$ is a uGC class \\
d) $\cal G$ has finite $V_\gamma$ dimension (finite $VC$ for 
indicator functions).
\end{thm}
}

%%%%%%%%%%%%%%%%%%%%%%%%%%%%%%%%%%%%%%%%%%%%%%%%%%%%%%%%%%%%%%%%%%%%%%

\foilhead{Two questions}

{\small 

This talk will answer the following two questions. \\ \\ \pause

\begin{itemize}

\item For ERM is there a notion of stability equivalent to
classical conditions (uGC) ? \\ \pause


\item Does this notion of stability ensure generalization for any symmetric algorithm ? \\  \pause

\end{itemize}

If (1) and (2) are true we will have a criteria that ensures generalization
for algorithms from SVMs to kNN. 





%%%%%%%%%%%%%%%%%%%%%%%%%%%%%%%%%%%%%%%%%%%%%%%%%%%%%%%%%%%%%%%%%%%%%%


\foilhead{Stability: continuity of $\cal A$}

Perturb the data $P: S \rightarrow S_p$ and see how much the functions
$f_S$ and $f_{S_p}$ differ. \pause \\

This can be construed as the continuity of $\cal A: S \rightarrow f_S$. \pause \\
 
Different definitions of stability amount to different norms or ``metrics''\\
$$\|f_S(x)-f_{S_p}(x)\|.$$


%%%%%%%%%%%%%%%%%%%%%%%%%%%%%%%%%%%%%%%%%%%%%%%%%%%%%%%%%%%%%%%%%%%%%%

\foilhead{Stability illustration}

\centerline{Ten samples}
\begin{center} 
  \includegraphics[scale=0.8]{c4.pts.eps}
\end{center}




%%%%%%%%%%%%%%%%%%%%%%%%%%%%%%%%%%%%%%%%%%%%%%%%%%%%%%%%%%%%%%%%%%%%%%

\foilhead{Stability illustration}

\centerline{Smoothest 10th degree polynomial}
\begin{center}
  \includegraphics[scale=0.8]{c4.ptsp10.eps}
\end{center}

%%%%%%%%%%%%%%%%%%%%%%%%%%%%%%%%%%%%%%%%%%%%%%%%%%%%%%%%%%%%%%%%%%%%%%

\foilhead{Stability illustration}

\centerline{Perturb data}
\begin{center}
  \includegraphics[scale=0.8]{c4.ptspert.eps}
\end{center}

%%%%%%%%%%%%%%%%%%%%%%%%%%%%%%%%%%%%%%%%%%%%%%%%%%%%%%%%%%%%%%%%%%%%%%

\foilhead{Stability illustration}

\centerline{Both polynomials}
\begin{center}
  \includegraphics[scale=0.8]{c4.ptspertp10.eps}
\end{center}


%%%%%%%%%%%%%%%%%%%%%%%%%%%%%%%%%%%%%%%%%%%%%%%%%%%%%%%%%%%%%%%%%%%%%%

\foilhead{Stability illustration}

\centerline{Restriction to 2nd degree polynomial}
\begin{center}
  \includegraphics[scale=0.8]{c4.ptsp2.eps}
\end{center}

%%%%%%%%%%%%%%%%%%%%%%%%%%%%%%%%%%%%%%%%%%%%%%%%%%%%%%%%%%%%%%%%%%%%%%

\foilhead{Stability illustration}


\centerline{Both polynomials}
\begin{center}
  \includegraphics[scale=0.8]{c4.ptspertp2.eps}
\end{center}



%%%%%%%%%%%%%%%%%%%%%%%%%%%%%%%%%%%%%%%%%%%%%%%%%%%%%%%%%%%%%%%%%%%%%%

\foilhead{The perturbation}

{\small

$S=\{z_1=(x_1,y_1),...,z_i=(x_i,y_i),...,z_n=(x_n,y_n)\}$, \\ \pause
$S^i=\{z_1=(x_1,y_1),...,z_{i-1}=(x_{i-1},y_{i-1}),z_{i+1}=(x_{i+1},y_{i+1}),...,z_n=(x_n,y_n)\}$. \\ \pause


When we talk about stability we will talk about the following two
functions
\begin{eqnarray*}
{\cal A}:& \, \,& S \rightarrow f_S \\
{\cal A}:& \, \,& S^i \rightarrow f_{S^i}. \\
\end{eqnarray*} 
}


%%%%%%%%%%%%%%%%%%%%%%%%%%%%%%%%%%%%%%%%%%%%%%%%%%%%%%%%%%%%%%%%%%%%%%


\foilhead{Uniform stability}

{\small 

Bousquet and Elisseeff introduced the following strong notion of
stability. \pause


\begin{defn}
$\cal A$ is distribution independent uniformly stable if
$$\forall S \in Z^n, \, \, \forall i \in\{1,...,n\}  \, \, \, \,
\sup_{z \in Z} |V(f_{S^i},z)-V(f_S,z)| \leq \beta_{U}^{(n)},$$
where $\beta_{U}^{(n)}$  goes to zero as $n \rightarrow \infty$. \\  \pause 
\end{defn}  


\begin{thm} Tikhonov regularization algorithms ${\cal A}_{reg}$ with bounded 
Lipschitz loss 
$$f_S = \arg \min_{f \in {\cal H}} \left[ n^{-1} \sum_{i=1}^n V(f,z_i) + \lambda \|f\|_{\cal H}^2 \right],$$
are uniformly stable with
$$\beta_{U}^{(n)} = \frac{C_1}{\lambda n}.$$
\end{thm}

%%%%%%%%%%%%%%%%%%%%%%%%%%%%%%%%%%%%%%%%%%%%%%%%%%%%%%%%%%%%%%%%%%%%%%


\foilhead{Uniformly stable algorithms generalize}
{\small

\begin{thm} If ${\cal A}$ is uniformly stable then with probability $1-\delta$ 
$$R[f_S] \leq R_S[f_S] + 2 \beta_U + (4n \beta_U + M)\sqrt{\frac{\ln{1/\delta}}{2 n}},$$
these bounds are meaningful if $\beta_U = o(n^{-1/2})$. 
\end{thm}
}




%%%%%%%%%%%%%%%%%%%%%%%%%%%%%%%%%%%%%%%%%%%%%%%%%%%%%%%%%%%%%%%%%%%%%%
\foilhead{Uniform stability is too strong}

{\small

The above theorem addresses algorithms like SVMs, ridge regression, splines,
and Parzen's windows. \pause \\ \\ 

\begin{countex}
ERM is not uniformly stable.  \pause

$X$ is uniform on $[0,1]$, $Y \in \{-1,1\}$, and we use the 
$\{0,1\}$-loss function  \pause \\
the target function $t(x) = {\pm 1}$  with probablity $1/2$ \\
the hypothesis space: ${\cal H} = \{f(x)= -1, f(x) = 1\}$. \\ \\ \pause
\end{countex}


For ERM: Uniform stability $\not \Leftrightarrow $ generalization.

}






%%%%%%%%%%%%%%%%%%%%%%%%%%%%%%%%%%%%%%%%%%%%%%%%%%%%%%%%%%%%%%%%%%%%%%

\foilhead{Cross-validation stability}


{\small


Devroye and Wagner introduced cross-validation stability in the
1970's to analyze kNN and kernel rules. \\ \pause

\begin{defn}
$\cal A$ is distribution independent $\mbox{CV}_{loo}$ stable if
$$\forall i \in\{1,...,n\} \, \, \, \, \forall \mu \, \, \, \,
\pr \left\{|V(f_{S^i},z_i)-V(f_S,z_i)| \leq \beta_{CV}^{(n)} \right\} \geq 1 - 
\delta_{CV}^{(n)},$$
where $\delta_{CV}^{(n)}$ and $\beta_{CV}^{(n)}$ goto zero as $n \rightarrow \infty$. 
\end{defn}


}


%%%%%%%%%%%%%%%%%%%%%%%%%%%%%%%%%%%%%%%%%%%%%%%%%%%%%%%%%%%%%%%%%%%%%%
\foilhead{Equivalence in the PAC realizable setting}

{\small

Kutin and Niyogi proved an equivalence in the PAC realizable setting (the target function is in the hypothesis space $\cal H$ of indicator functions). \pause \\ 

\begin{thm} 
For ERM in the PAC realizable setting the following
are equivalent \\
a) ${\cal H}$ has finite VC dimension \\
b) ERM is consistent \\
c) $\cal A$ is $\mbox{CV}_{loo}$ stable with $\beta_{CV} = O(n^{-1})$ and
$\delta_{CV} = \Omega(e^{-n})$. \pause \\ 
\end{thm}


Is there an equivalence beyond the PAC realizable setting ?
}



%%%%%%%%%%%%%%%%%%%%%%%%%%%%%%%%%%%%%%%%%%%%%%%%%%%%%%%%%%%%%%%%%%%%%%

\foilhead{For ERM $\mbox{CV}_{loo}$ is equivalent to consistency}


{\small

\begin{thm}
If the loss function is bounded $0 \leq V(f,z) \leq M$ then the following
statements are equivalent for ERM  \\
a) ERM is consistent \\
b) $\cal G$ is a uGC class \\
c) $\cal G$ has finite $V_\gamma$ dimension (finite $VC$ for 
indicator functions) \\ \pause
d) $\cal A$ is distribution-independent $\mbox{CV}_{loo}$ stable.
\end{thm}

}


%%%%%%%%%%%%%%%%%%%%%%%%%%%%%%%%%%%%%%%%%%%%%%%%%%%%%%%%%%%%%%%%%%%%%%


\foilhead{What about non-ERM algorithms}

{\small 


\begin{countex} 
$\mbox{CV}_{loo}$ is not sufficient for generalization of non-ERM
algorithms. \\ \pause


$X$ is uniform on $[0,1]$, $Y \in \{-1,1\}$, and we use the 
$\{0,1\}$-loss function  \\
the target function $t(x) = 1$.  \pause \\
The following non-ERM algorithm is $\mbox{CV}_{loo}$ stable
\begin{displaymath}
f_S(x) = \left\{\begin{array}{ll}
                        (-1)^n & \mbox{if n is a training point} \\
                        (-1)^{n+1} & \mbox{otherwise}
                                                   .    \end{array}
                                                \right. \pause  
\end{displaymath}  
This algorithm does not generalize.
\end{countex}
}

%%%%%%%%%%%%%%%%%%%%%%%%%%%%%%%%%%%%%%%%%%%%%%%%%%%%%%%%%%%%%%%%%%%%%%

\foilhead{Learning stability}


{\small

\begin{defn}
$\cal A$ is distribution independent learning stable if \\
a) it is $\mbox{CV}_{loo}$ stable \\ \pause
b) it is error stable
$$\forall i \in\{1,...,n\} \, \, \, \, \forall \mu \, \, \, \,
\pr \left\{|R[f_{S^i}]-R[f_S]| \leq \beta_{Er}^{(n)} \right\} \geq 1 - 
\delta_{Er}^{(n)},$$
where $\delta_{Er}^{(n)}$ and $\beta_{Er}^{(n)}$ goto zero as $n \rightarrow \infty$, \\ \pause
c) it is empirical error stable 
$$\forall i \in\{1,...,n\} \, \, \, \, \forall \mu \, \, \, \,
\pr \left\{|R_S[f_{S^i}]-R_S[f_S]| \leq \beta_{E}^{(n)} \right\} \geq 1 - 
\delta_{E}^{(n)},$$
where $\delta_{E}^{(n)}$ and $\beta_{E}^{(n)}$ goto zero as $n \rightarrow \infty$. \\ \pause
\end{defn}

ERM implies (c), and ERM on a uGC class implies (b) so
learning stability $\Leftrightarrow$  consistency of ERM.


}




%%%%%%%%%%%%%%%%%%%%%%%%%%%%%%%%%%%%%%%%%%%%%%%%%%%%%%%%%%%%%%%%%%%%%%

\foilhead{Learning stability implies generalization}


{\small

\begin{thm} If $\cal A$ is symmetric and learning stable then with probability
$1-\delta_{gen}$
$$|R[f_S] - R_S[f_S]| \leq \beta_{gen},$$ \pause
where 
$$\delta_{gen} = \beta_{gen} = (2M \beta_{CV} + 2 M^2 \delta_{CV} + 3 M \beta_{Er} + 3 M^2 \delta_{Er} + 5 M \beta_{E} + 5 M^2 \delta_{E})^{1/4}.$$
\end{thm} 
}




%%%%%%%%%%%%%%%%%%%%%%%%%%%%%%%%%%%%%%%%%%%%%%%%%%%%%%%%%%%%%%%%%%%%%%

\foilhead{Summary}


{\small

\begin{thm}
For symmetric $\cal A$ with bounded loss \\ \pause
a) learning stability implies generalization \\ \pause
b) for ERM learning stability is equivalent to consistency.   
\end{thm} 

}


%%%%%%%%%%%%%%%%%%%%%%%%%%%%%%%%%%%%%%%%%%%%%%%%%%%%%%%%%%%%%%%%%%%%%%

\foilhead{A conjecture}


{\small

\begin{conj} 
$\mbox{CV}_{loo}$ stability and empirical error stability is sufficient
for the generalization of symmetric algorithms. 
\end{conj} 

}

%%%%%%%%%%%%%%%%%%%%%%%%%%%%%%%%%%%%%%%%%%%%%%%%%%%%%%%%%%%%%%%%%%%%%%

\foilhead{Generalization given concentration}


{\small

A natural way of proving generalization is using concentration. \\
\begin{thm}
A symmetric algorithm $\cal A$ generalizes if\\
a) is distribution independent $\mbox{CV}_{loo}$ stable \\ \pause
b) has concentration of empirical and expected error: with
probability $1-\delta^{(n)}$
\begin{eqnarray*}
|R[f_S]-\E_S R[f_S]| & \leq & \tau_{Er}^{(n)} \\
|R_S[f_S]-\E_S R_S[f_S]| & \leq & \tau_{E}^{(n)},
\end{eqnarray*}
where $\delta^{(n)}$, $\tau_{E}^{(n)}$, and $\tau_{Er}^{(n)}$ goto zero 
as $n \rightarrow \infty$. 
\end{thm} 
}


%%%%%%%%%%%%%%%%%%%%%%%%%%%%%%%%%%%%%%%%%%%%%%%%%%%%%%%%%%%%%%%%%%%%%%

\foilhead{Generalization without concentration}


{\small


\begin{countex} 
There are algorithms that are learning stable and generalize but do not 
concentrate. \\ \pause


$X$ is uniform on $[0,1]$, $Y \in \{-1,1\}$, and we use the 
$\{0,1\}$-loss function  \\
the target function $t(x) = 1$.  \pause \\
We examine the following algorithm
\begin{displaymath}
f_S(x) = \left\{\begin{array}{ll}
                        -1 & \mbox{if $\lceil \frac{n}{2} \rceil$ samples are lager than .5} \\
                        1 & \mbox{otherwise}
                                                   .    \end{array}
                                                \right.  \pause
\end{displaymath}  
The expected error of this algorithm is $.5$ and therefore does not 
concentrate. However, this algorithm does generalize and is learning stable.
\end{countex}
}



%%%%%%%%%%%%%%%%%%%%%%%%%%%%%%%%%%%%%%%%%%%%%%%%%%%%%%%%%%%%%%%%%%%%%%

\foilhead{An alternative decomposition}


{\small

Devroye and Wagner used the following decomposition to prove
kNN and kernel rules generalize.


\begin{thm}
If 
$$\lim_{n \rightarrow \infty} \E_S(R[f_S]-R_S[f_S])^2 = 0,$$
then $\cal A$ generalizes. \\ \\ \pause
The following decomposition holds
$$\E_S(R[f_S]-R_S[f_S])^2 \leq  \Ig{2 \, \E_S (R[f_S]-R_{loo}[S])^2} +  \Ie{2 \, \E_S (R_{loo}[S]-R_S[f_S])^2}.$$ \\ \pause 
\end{thm}

$\mbox{CV}_{loo}$ stability was used to control the second term 
$$\Ie{\E_S (R_{loo}[S]-R_S[f_S])^2 \leq 2 M \beta_{CV} + 2 M^2 \delta_{CV}}.$$

}

%%%%%%%%%%%%%%%%%%%%%%%%%%%%%%%%%%%%%%%%%%%%%%%%%%%%%%%%%%%%%%%%%%%%%%



\foilhead{Hypothesis stability}


{\small

The second term was controlled by hypothesis stability
$$\Ig{\E_S (R[f_S] - R_{loo})^2 \leq 2 M \beta_{H} + 2 M^2 \delta_{H}}.$$ \pause \\


\begin{defn}
$\cal A$ is distribution independent $\mbox{H}_{loo}$ stable if
$$\forall i \in\{1,...,n\} \, \, \, \, \forall \mu \, \, \, \,
\pr \left\{\E_z|V(f_{S^i},z)-V(f_S,z)| \leq \beta_{H}^{(n)} \right\} \geq 1 - 
\delta_{H}^{(n)},$$
where $\delta_{H}^{(n)}$ and $\beta_{H}^{(n)}$ goto zero as $n \rightarrow \infty$.
\end{defn}

}


%%%%%%%%%%%%%%%%%%%%%%%%%%%%%%%%%%%%%%%%%%%%%%%%%%%%%%%%%%%%%%%%%%%%%%



\foilhead{Is ERM hypothesis stable ?}


{\small
This is an open question. \\ \pause

ERM is hypothesis stable under the following conditions:

\begin{enumerate}

\item The realizable setting and $\cal H$ is uGC  \pause

\item Convex loss and $\cal H$ is uGC and convex \pause

\item $\cal H$ is uGC and there is a unique best function \pause

\item $|{\cal H}|$ is finite. 


\end{enumerate}



}

%%%%%%%%%%%%%%%%%%%%%%%%%%%%%%%%%%%%%%%%%%%%%%%%%%%%%%%%%%%%%%%%%%%%%%



\foilhead{What algorithms are learning stable}


{\small

The following is a list of symmetric algorithms that we know are learning stable: \pause

\begin{enumerate}

\item kNN with $k \rightarrow \infty$ and $\frac{k}{n} \rightarrow 0$  \pause

\item kernel rules with the width $h_n \rightarrow 0$ and 
$h_n n \rightarrow \infty$ \pause

\item a variety of bagging algorithms \pause

\item regularization algorithms \pause

\item ERM on a uGC class \pause

\item online regularized least squares.




\end{enumerate}



}

%%%%%%%%%%%%%%%%%%%%%%%%%%%%%%%%%%%%%%%%%%%%%%%%%%%%%%%%%%%%%%%%%%%%%%



\foilhead{Learning languages}


{\small

The language learning algorithm ${\cal A}_L: {\cal Se} \rightarrow 
{\cal Gr}$. \\ \pause

For ${\cal A}_L$ there exists a class $H_{{\cal A}_L}$ which is the class
of all learnable grammars.  \\ \pause

In generative linguistics the class of grammars $H_{{\cal A}_L}$ is 
called Universal Grammar and different linguistic
theories attempt to characterize the nature of this class. \\ \pause

In many cases, ${\cal A}_L$ may admit an easier mathematical 
characterization than ${\cal H}_{{\cal A}_L}$.



}


%%%%%%%%%%%%%%%%%%%%%%%%%%%%%%%%%%%%%%%%%%%%%%%%%%%%%%%%%%%%%%%%%%%%%%





\foilhead{Open problems and comments}


{\small


\begin{enumerate}

\item  The rates for the stability approach do not seem to match
up with the rates using using VC-type bounds. \pause

\item We have a direct proof of the following statement:\\
{\it If  $\cal H$ has infinite VC dimension, then $\forall n,
\beta_{CV} > \frac{1}{8}$}. \pause 

\item Is ERM hypothesis stable ? \pause

\item Are all three stability terms needed for generalization. \pause

\item Similar results for nonsymmteric algorithms, for example online algorithms. \pause

\item Designing stable algorithms. \pause

\item Martingale inequalitites with bad sets. 

\end{enumerate}



}





%%%%%%%%%%%%%%%%%%%%%%%%%%%%%%%%%%%%%%%%%%%%%%%%%%%%%%%%%%%%%%%%%%%%%%

\foilhead{Proof sketch}

{\small

For ERM
$$V(f_{S^i},z_i) - V(f_S,z_i) \geq 0.$$ \pause

\begin{eqnarray*} 
(\beta,\delta)-\mbox{$\mbox{CV}_{loo}$ stability } & \Leftrightarrow & \lim_{n\rightarrow \infty} \E_S[|V(f_{S^i},z_i) - V(f_S,z_i)|] = 0, \\ \pause
&\Leftrightarrow & \lim_{n\rightarrow \infty} \E_S[V(f_{S^i},z_i) - V(f_S,z_i)] = 0, \\ \pause
& \Leftrightarrow & \lim_{n\rightarrow \infty} \E_S R[f_{S^i}] - \E_S R_S[f_S] = 0, \\ \pause
& \Leftrightarrow &\lim_{n\rightarrow \infty} \E_S R[f_{S}] = \lim_{n\rightarrow \infty} \E_S R_S[f_S], \\ \pause
& \Leftrightarrow & \mbox{Consistency of ERM}. \\
\end{eqnarray*}

}


%%%%%%%%%%%%%%%%%%%%%%%%%%%%%%%%%%%%%%%%%%%%%%%%%%%%%%%%%%%%%%%%%%%%%%

\foilhead{Proof sketch}


{\small

If we can control: \\
$$\lim_{n \rightarrow \infty} \E_S(R[f_S] - R_S[f_S])^2 = 0,$$
then we can use the Bienaym\'{e}-Chebyshev inequality for the result. \pause

\begin{eqnarray*}
\E_S(R[f_S]-R_S[f_S])^2 & = & \E_S( R[f_S]^2 +R_S[f_S]^2 -2 R[f_S] R_S[f_S]) \\
& = & \E_S[R[f_S](R[f_S]-R_S[f_S])] + \E_S[R_S[f_S](R_S[f_S]-R[f_S])]. 
\end{eqnarray*}

Using learning stability we can upper bound the two terms above. Zero
is a trivial lower bound.
}



\end{document}










